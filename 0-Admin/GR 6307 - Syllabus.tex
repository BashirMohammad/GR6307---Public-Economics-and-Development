\documentclass[11pt]{article}
\usepackage{mathpazo}
\renewcommand{\sfdefault}{lmss}
\renewcommand{\ttdefault}{lmtt}
\usepackage[T1]{fontenc}
\usepackage[utf8]{inputenc}
\usepackage{setspace}
\onehalfspacing
\usepackage{hyperref}
\hypersetup{colorlinks=true,linkcolor=[rgb]{0,0,0.6},citecolor=[rgb]{0,0,0.6},urlcolor=[rgb]{0,0,0.6}}
%\usepackage[usenames,dvipsnames]{xcolor}
\DeclareSymbolFont{operators}   {OT1}{lmr} {m}{n}
\DeclareSymbolFont{letters}     {OML}{cmm} {m}{it}
\DeclareSymbolFont{symbols}     {OMS}{cmsy}{m}{n}
   \SetSymbolFont{operators}{bold}{OT1}{cmr}{b}{n}
   \SetSymbolFont{letters}{bold}{OML}{cmm}{b}{it}
   \SetSymbolFont{symbols}{bold}{OMS}{cmsy}{b}{n}

\usepackage[style=chicago-authordate,maxnames=10]{biblatex}

\addbibresource{readinglist.bib}
\begin{document}

\title{ECON GR6307: \\ Public Economics and Development \\ Reading List}
\author{Michael Carlos Best}

\maketitle

Readings in bold and marked with ** are \textit{absolutely} required readings. Readings marked with *(M/DD) are the summary readings for that week's lecture. Each week you should prepare a 6-slide summary (excluding the title slide) of the paper and bring it with you on a USB drive. I will randomly call on someone to present their summary each week. NB the randomization will be with replacement, so don't think that you can't be called more than once! The summary should take the form of a general overview outlining why the paper is interesting/important, how the authors approached their question, the methods/data they used, and their main results (copy the tables into your slides).

\section{Applied Welfare Analysis}

In public economics we often want to make normative statements about the social desirability of policies, and quantify tradeoffs between policies. This section presents the modern tools public economists use to do this. 

For 1/24, please read Hendren \& Sprung-Keyser (2020). Instead of preparing a summary of the paper (I'll go over it in the lecture), pick a policy that interests you and go through all the steps of how its MVPF is calculated. You can find all the ones in the paper plus a whole lot more that people have worked out MVPFs for since then at \url{https://www.policyimpacts.org/policy-impacts-library}. Then prepare slides on that, with your own take on the ingredients of the MVPF.

\begin{enumerate}
\item \fullcite{Chetty2009}.
\item \textbf{\fullcite{Chetty2009ARE}.$^{**}(1/24)$}
\item \fullcite{Feldstein1999}.
\item \textbf{\fullcite{Finkelstein2019}.$^{**}(1/17)$}
\item \fullcite{Kleven2018}.
\item \fullcite{Hendren2016}.
\item \fullcite{Hendren2020}.
\item \fullcite{HendrenSprungKeyser2019}.$^{*}(1/24)$
\item \fullcite{HendrenSprungKeyser2022}.
\item \fullcite{SaezStantcheva2016}
\item \fullcite{SlemrodYitzhaki1996}
\item \fullcite{SlemrodYitzhaki2001}
\end{enumerate}


\section{Taxation}
Tax systems in rich countries look very different from those in poor countries. How should tax systems be designed in the presence of high levels of tax evasion and informality? How much tax evasion is there? How can governments reduce tax evasion?

\subsection{Taxation in Developing Countries: Macro Perspective}

\begin{enumerate}
\item \fullcite{BachasFisherPostJensenZucman2023}.$^{*}(1/31)$
\item \fullcite{BesleyPersson2009}.
\item \fullcite{BesleyPersson2011}.
\item \textbf{\fullcite{BesleyPersson2013}.$^{**}(1/31)$}
\item \fullcite{BoadwaySato2009}.
\item \fullcite{BurgessStern}.
\item \fullcite{GordonLi2009}.
\item \fullcite{LaPortaShleifer2014}.
\end{enumerate}

\subsection{Tax Evasion: Theory and Evidence from Rich Countries}

\begin{enumerate}
\item \fullcite{AdvaniEtAl2020}.
\item \fullcite{AllinghamSandmo}.
\item \fullcite{Alstadsaeteretal2018}.
\item \fullcite{Alstadsaeteretal2018b}.
\item \fullcite{Andreonietal1998}.
\item \textbf{\fullcite{Artavanisetal2016}.$^{**}(2/7)$}
\item \fullcite{BoningHendrenSprungKeyserStewart2023}
\item \fullcite{Dwengeretal2016}.
\item \fullcite{GuytonEtAl2021}.$^{*}(2/7)$
\item \fullcite{Klevenetal2011}.
\item \fullcite{Slemrod2007}.
\item \fullcite{Slemrod2017}.
\item \fullcite{SlemrodYitzhaki}.
\end{enumerate}

\subsection{Empirical Evidence on Taxation in Low- and Middle-Income Countries}


\begin{enumerate}
   \item \fullcite{AmodioChoiDeGiorgiRahman2018}.
   \item \fullcite{AvilaLondono2020}.
   \item \fullcite{BalanBergeronTourekWeigel2021}.
   \item \textbf{\fullcite{BachasGadenneJensen2020}.$^{**}(2/14)$}
   \item \fullcite{BachasSoto2018}.
   \item \fullcite{BenhassineEtal2018}.
   \item \fullcite{BrockmeyerHernandez2018}.
   \item \fullcite{ChenLiuSuarezXu2018}.
   \item \fullcite{GadenneNandiRathelot2020}.
   \item \fullcite{Gorodnichenkoetal2009}.
   \item \fullcite{Jensen2019}.
   \item \fullcite{LaPortaShleifer2014}.
   \item \fullcite{Naritomi2018}.$^{*}(2/14)$
   \item \fullcite{Pomeranz2015}.
   \item \fullcite{SlemrodUrRehmanWaseem2020}.
   \item \fullcite{Ulyssea2018}.
   \item \fullcite{Weigel2020}.
\end{enumerate}

\subsection{Tax Policy and Tax Administration}
\begin{enumerate}
   \item \fullcite{BasriFelixHannaOlken2020}.
   \item \fullcite{BergeronTourekWeigel2021}.$^{*}(2/21)$
   \item \textbf{\fullcite{Bestetal2015}.$^{**}(2/21)$}
   \item \fullcite{BrockmeyerEtAl2021}.
   \item \fullcite{KeenSlemrod2017}.
\end{enumerate}


\subsection{International Taxation and Developing Countries}

\begin{enumerate}
\item \fullcite{AndersenJohannesenRijkers2020}.
\item \fullcite{ChalendardFernandesRaballandRijkers2020}.
\item \fullcite{FismanWei2004}.
\item \fullcite{Sequeira2016}.
\item \fullcite{Suarez2018}.
\item \fullcite{TorslovWierZucman2018}.
\item \fullcite{Zucman2013}.
\item \fullcite{Zucman2014}.
\end{enumerate}

\section{ Anti-poverty Programs}
Targeted transfers to poor household are a huge part of government spending in low- and middle-income countries. How should these programs be designed? Should they be monetary or in-kind transfers? Should they be means-tested? If so, how will eligibility be determined?

% add "encouragement and distortionary effects of conditional cash transfers by Bryan, Chowdhury Mobarak Morten & Smits.

\subsection{Theory}

\begin{enumerate}
\item \fullcite{Akerlof1978}.
\item \fullcite{Bertrandetal2013}.
\item \fullcite{BesleyCoate1992}.$^{*(2/28)}$
\item \fullcite{BrewerSaezShephard2010}.
\item \fullcite{KlevenKopczuk2011}.
\item \textbf{\fullcite{NicholsZeckhauser1982}.$^{**(2/28)}$}
\item \fullcite{Saez2002}.
\end{enumerate}

\subsection{Evidence from Rich Countries}

\begin{enumerate}
\item \fullcite{BrewerSaezShephard2010}.
\item \fullcite{Chettyetal2013}.
\item \fullcite{DeshpandeLi2017}.$^{*(3/6)}$
\item \textbf{\fullcite{FinkelsteinNoto2019}.$^{**(3/6)}$}
\item \fullcite{Rothstein2010}.
\item \fullcite{Saez2006}.
\end{enumerate}

\subsection{Targeting in Developing Countries}

\begin{enumerate}
\item \fullcite{Alatasetal2012}.
\item \textbf{\fullcite{Alatasetal2016}.$^{**(3/27)}$}
\item \fullcite{AngelucciDeGiorgi2009}.
\item \fullcite{AttanasioLechene2014}.
\item \fullcite{Banerjeeetal2018}.
\item \fullcite{BanerjeeHannaOlkenSumarto2018}.
\item \fullcite{BanerjeeHannaOlkenLisker2022}.
\item \fullcite{Barnwal2017}.
\item \fullcite{Basurtoetal2017}.
\item \fullcite{Brolloetal2016}.
\item \fullcite{CahyadiHannaOlkenPrimaSatriawanSyamsulhakim2018}.
\item \fullcite{Cohenetal2015}.$^{*(3/27)}$
\item \fullcite{HannaOlken2018}.
\item \fullcite{HaushoferNiehausParamoMiguelWalker2022}.
\item \fullcite{ImbertPapp2015}.
\item \fullcite{ParkerTodd2017}.
\end{enumerate}

\subsection{Transfer Design}

\begin{enumerate}
\item \fullcite{BairdMcIntoshOzler2011}.
\item \fullcite{BalboniEtAl2021}.
\item \fullcite{BanerjeeDufloSharma2021}.
\item \fullcite{BanerjeeEtAl2021}.
\item \fullcite{BergstromDodds2021}.
\item \fullcite{BryanEtAl2021}.
\item \textbf{\fullcite{Cunhaetal2017}.$^{**(4/3)}$}
\item \fullcite{EggerEtAl2021}.$^{*(4/3)}$
\item \fullcite{GadenneNorrisSinghalSukhtankar2022}.
\item \fullcite{GerardNaritomiSilva2021}.
\end{enumerate}

\section{The Personnel Economics of the State}
The government is the largest employer in most countries, but public service delivery is notoriously inefficient. How can governments attract honest, capable and motivated workers? How will the government monitor and incentivize their workers?

%add: Russia and Punjab. Don't do the Banerjee paper, instead do our model or maybe the Guriev one? fred's new decentralization paper? Oriana's new zambia paper. Oriana's targeting and social networks paper in Uganda?
%Add in the edu and manu's patronage paper, and Diana's too?

\subsection{Theory}

\begin{enumerate}
\item \fullcite{AghionTirole1997}
\item \fullcite{Banerjeeetal2013}.
\item \textbf{\fullcite{BenabouTirole2006}}.$^{**(4/10)}$
\item \fullcite{BesleyGhatak2005}
\item \fullcite{LazearOyer2012}
\item \fullcite{OlkenPande2012}
\item \fullcite{Prendergast2003}
\item \fullcite{Prendergast2007}
\item \fullcite{ShleiferVishny1993}.
\end{enumerate}

\subsection{Financial Incentives}

\begin{enumerate}
\item \fullcite{BertrandEtAl2007}.$^{*(4/10)}$
\item \fullcite{Chaudhuryetal2006}
\item \fullcite{DeReeetal2017}.$^{*(4/17)}$
\item \fullcite{Dufloetal2013}
\item \fullcite{Dufloetal2012}
\item \fullcite{Finanetal2017}
\item \fullcite{KhanKhwajaOlken2016}
\item \textbf{\fullcite{MuralidharanSundararaman2011}}.$^{**(4/17)}$
\item \fullcite{Olken2007}
\item \fullcite{OlkenPande2012}
\end{enumerate}

\subsection{Non-financial Incentives}

\begin{enumerate}
\item \fullcite{Ashrafetal2014}.
\item \fullcite{Bandieraetal2021}.
\item \fullcite{Bestetal2023}
\item \fullcite{Callenetal2015}
\item \textbf{\fullcite{Dufloetal2016}}.$^{**(4/24)}$
\item \fullcite{KhanKhwajaOlken2017}
\item \fullcite{Khan2023}.$^{*(4/24)}$
\end{enumerate}

\subsection{Recruitment \& Selection}

\begin{enumerate}
\item \fullcite{Ashrafetal2016}.
\item \fullcite{BrownAndrabi2021}.
\item \textbf{\fullcite{DalBoetal2013}}.$^{**(5/1)}$
\item \fullcite{Deserrano2017}.
\item \fullcite{LeaverEtAl2022}.
\item \fullcite{Xu2018}.$^{*(5/1)}$
\end{enumerate}

\section{Data \& Technology in Government}
Most policy problems involve prediction of a counterfactual (what if we raise tax rates?) or a state of the world (how much poverty is there?). How can machine learning methods help governments make these predictions? Can new technologies be used to monitor government workers and increase their productivity and/or effort?

\begin{enumerate}
\item \fullcite{Abelsonetal2014}
\item \fullcite{Ackermannetal2016}
\item \fullcite{BanerjeeHannaOlkenKyleSumarto2016}
\item \fullcite{Banerjeeetal2017}.
\item \fullcite{Blumenstocketal2015}
\item \fullcite{Blumenstock2016}
\item \fullcite{CallenLong2015}
\item \fullcite{Callenetal2016}
\item \fullcite{CasaburiTroiano2016}
\item \fullcite{Chalfinetal2016}
\item \fullcite{Engstrometal2016}
\item \fullcite{Fujiwara2015}
\item \fullcite{Glaeseretal2015}
\item \fullcite{Glaeseretal2016}.
\item \fullcite{Goeletal2016}
\item \fullcite{Jeanetal2016}
\item \fullcite{Kangetal2013}
\item \textbf{\fullcite{Kleinbergetal2015}}
\item \textbf{\fullcite{Kleinbergetal2017}}
\item \fullcite{Muralidharanetal2016}.
\item \fullcite{Muralidharanetal2017}
\item \fullcite{MuralidharanSinghGanimian2017}
\end{enumerate}


%\printbibliography
\end{document}