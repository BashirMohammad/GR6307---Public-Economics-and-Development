\documentclass[11pt]{article}
\usepackage{mathpazo}
\renewcommand{\sfdefault}{lmss}
\renewcommand{\ttdefault}{lmtt}
\usepackage[T1]{fontenc}
\usepackage[utf8]{inputenc}
\usepackage{setspace}
\onehalfspacing
\usepackage{hyperref}
\hypersetup{colorlinks=true,linkcolor=[rgb]{0,0,0.6},citecolor=[rgb]{0,0,0.6},urlcolor=[rgb]{0,0,0.6}}
%\usepackage[usenames,dvipsnames]{xcolor}
\DeclareSymbolFont{operators}   {OT1}{lmr} {m}{n}
\DeclareSymbolFont{letters}     {OML}{cmm} {m}{it}
\DeclareSymbolFont{symbols}     {OMS}{cmsy}{m}{n}
   \SetSymbolFont{operators}{bold}{OT1}{cmr}{b}{n}
   \SetSymbolFont{letters}{bold}{OML}{cmm}{b}{it}
   \SetSymbolFont{symbols}{bold}{OMS}{cmsy}{b}{n}

\usepackage[style=chicago-authordate,maxnames=10]{biblatex}

\addbibresource{readinglist.bib}
\begin{document}

\title{ECON GR6307: \\ Public Economics and Development \\ Reading List}
\author{Michael Carlos Best}

\maketitle

Readings in bold and marked with ** are \textit{absolutely} required readings. Readings marked with *(M/DD) are the summary readings for that week's lecture. Each week you should pick one of these papers and prepare a 6-slide summary (excluding the title slide) of the paper and bring it with you on a USB drive. I will randomly call on someone to present their summary each week. NB the randomization will be with replacement, so don't think that you can't be called more than once! The summary should take the form of a general overview outlining why the paper is interesting/important, how the authors approached their question, the methods/data they used, and their main results (copy the tables into your slides).

\section{Taxation}
Tax systems in rich countries look very different from those in poor countries. How should tax systems be designed in the presence of high levels of tax evasion and informality? How much tax evasion is there? How can governments reduce tax evasion?

\begin{enumerate}
\item \fullcite{AllinghamSandmo}.
\item \fullcite{Alstadsaeteretal2017}.$^{*(1/24)}$
\item \fullcite{Andreonietal1998}.
\item \textbf{\fullcite{Artavanisetal2016}.$^{**}$}
\item \fullcite{BenhassineEtal2018}.
\item \fullcite{BesleyPersson2009}.
\item \fullcite{BesleyPersson2011}.
\item \fullcite{BesleyPersson2013}.
\item \fullcite{Bestetal2015}.$^{*(1/31)}$
\item \fullcite{BoadwaySato2009}.
\item \fullcite{BurgessStern}.
\item \fullcite{Chetty2009}.
\item \fullcite{Dwengeretal2016}.
\item \fullcite{FismanWei2004}.
\item \textbf{\fullcite{GordonLi2009}.$^{**}$}
\item \fullcite{Gorodnichenkoetal2009}.
\item \fullcite{Jensen2016}.$^{*(1/24)}$
\item \fullcite{Klevenetal2011}.
\item \fullcite{LaPortaShleifer2014}.
\item \fullcite{LuttmerSinghal2014}.
\item \textbf{\fullcite{Pomeranz2015}.$^{**}$}
\item \fullcite{Sequeira2016}.$^{*(1/31)}$
\item \fullcite{Slemrod2007}.
\item \fullcite{SlemrodYitzhaki}.
\item \fullcite{Zucman2013}.
\item \fullcite{Zucman2014}.
\end{enumerate}

\section{ Anti-poverty Programs}
Targeted transfers to poor household are a huge part of government spending in low- and middle-income countries. How should these programs be designed? Should they be monetary or in-kind transfers? Should they be means-tested? If so, how will eligibility be determined?

\begin{enumerate}
\item \fullcite{Akerlof1978}.
\item \fullcite{Alatasetal2012}.
\item \fullcite{Alatasetal2016}.
\item \fullcite{AngelucciDeGiorgi2009}.
\item \fullcite{AttanasioLechene2014}.
\item \fullcite{BairdMcIntoshOzler2011}.
\item \fullcite{Banerjeeetal2018}.
\item \fullcite{Barnwal2017}.
\item \fullcite{Basurtoetal2017}.
\item \fullcite{Bertrandetal2013}.
\item \fullcite{BesleyCoate1992}.
\item \fullcite{BrewerSaezShephard2010}.
\item \fullcite{Brolloetal2016}.
\item \fullcite{Chettyetal2013}.
\item \fullcite{Cohenetal2015}.
\item \fullcite{Cunhaetal2017}
\item \fullcite{DeshpandeLi2017}.
\item \fullcite{Dupasetal2016}.
\item \fullcite{GahvariMattos2007}.
\item \fullcite{ImbertPapp2015}.
\item \fullcite{KlevenKopczuk2011}.
\item \fullcite{NicholsZeckhauser1982}.
\item \fullcite{ParkerTodd2017}.
\item \fullcite{Rothstein2010}.
\item \fullcite{Saez2002}.
\item \fullcite{Saez2006}.
\end{enumerate}

\section{The Personnel Economics of the State}
The government is the largest employer in most countries, but public service delivery is notoriously inefficient. How can governments attract honest, capable and motivated workers? How will the government monitor and incentivize their workers?

\begin{enumerate}
\item \fullcite{AghionTirole1997}
\item \fullcite{Ashrafetal2014}
\item \fullcite{Ashrafetal2016}
\item \fullcite{Banerjeeetal2013}
\item \fullcite{BenabouTirole2006}
\item \fullcite{BesleyGhatak2005}
\item \fullcite{Callenetal2015}
\item \fullcite{Chaudhuryetal2006}
\item \fullcite{DalBoetal2013}
\item \fullcite{DeReeetal2017}
\item \fullcite{Deserrano2017}
\item \fullcite{Dufloetal2013}
\item \fullcite{Dufloetal2016}
\item \fullcite{Dufloetal2012}
\item \fullcite{Finanetal2017}
\item \fullcite{KhanKhwajaOlken2016}
\item \fullcite{KhanKhwajaOlken2017}
\item \fullcite{LazearOyer2012}
\item \fullcite{MuralidharanSundararaman2011}
\item \fullcite{Olken2007}
\item \fullcite{OlkenPande2012}
\item \fullcite{Prendergast2003}
\item \fullcite{Prendergast2007}
\end{enumerate}

\section{Data \& Technology in Government}
Most policy problems involve prediction of a counterfactual (what if we raise tax rates?) or a state of the world (how much poverty is there?). How can machine learning methods help governments make these predictions? Can new technologies be used to monitor government workers and increase their productivity and/or effort?

\begin{enumerate}
\item \fullcite{Abelsonetal2014}
\item \fullcite{Ackermannetal2016}
\item \fullcite{BanerjeeHannaOlkenKyleSumarto2016}
\item \fullcite{Banerjeeetal2017}
\item \fullcite{Blumenstocketal2015}
\item \fullcite{Blumenstock2016}
\item \fullcite{CallenLong2015}
\item \fullcite{Callenetal2016}
\item \fullcite{CasaburiTroiano2016}
\item \fullcite{Chalfinetal2016}
\item \fullcite{Engstrometal2016}
\item \fullcite{Fujiwara2015}
\item \fullcite{Glaeseretal2015}
\item \fullcite{Glaeseretal2016}
\item \fullcite{Goeletal2016}
\item \fullcite{Jeanetal2016}
\item \fullcite{Kangetal2013}
\item \fullcite{Kleinbergetal2015}
\item \fullcite{Kleinbergetal2017}
\item \fullcite{Muralidharanetal2016}
\item \fullcite{Muralidharanetal2017}
\item \fullcite{MuralidharanSinghGanimian2017}
\item \fullcite{Naritomi2016}
\end{enumerate}


%\printbibliography
\end{document}