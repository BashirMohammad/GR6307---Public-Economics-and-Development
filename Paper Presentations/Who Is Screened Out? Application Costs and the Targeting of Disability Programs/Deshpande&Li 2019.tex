\documentclass{beamer}

\usepackage[utf8]{inputenc} % Ensure proper encoding
\usepackage{graphicx} % For including images
\usepackage{hyperref} % For hyperlinks
\usepackage{amsmath,amsfonts,amssymb} % For mathematical symbols and equations
\begin{document}

\title{Who Is Screened Out? Application Costs and the Targeting of Disability Programs}
\author{Manasi Deshpande \& Yue Li}
\date{\today}

\frame{\titlepage}


\begin{frame}
    \frametitle{Introduction}
    \begin{itemize}
        \item Social Security Disability Insurance (SSDI or DI) provides cash benefits and Medicare to disabled workers in the United States. 
        \item In 2015, it covered nearly 9 million workers.
        \item  Additionally, Supplemental Security Income (SSI) offers cash welfare and Medicaid eligibility to nearly 7 million low-income, disabled Americans, including 1.4 million children in 2015.
        \item This program strictly targets those with severe disabilities who are unable to work.
    \end{itemize}
\end{frame}

\begin{frame}
    \frametitle{This Paper}
    \begin{itemize}
        \item This paper examines the role of application costs in the targeting of disability programs and importantly looks at trade-offs between screening through ordeal and targeting.
        \item The authors use a novel dataset of 1.6 million initial disability applications to estimate the effect of application costs on the composition of program participants using a natural experiment that lead to closing of some field offices that assisted in the application process. 
        \item They find that application costs are a key determinant of program participation, and that the targeting of disability programs is sensitive to the stringency of the application process.
        
    \end{itemize}
\end{frame}

\begin{frame}[allowframebreaks]
    \frametitle{Institutional Background and Data}
    \begin{itemize}
        \item The Social Security Administration (SSA) administers both SSDI and SSI. 
        \item The application process is complex and time-consuming, requiring extensive documentation and medical evidence.
        \item The authors use a novel dataset of 1.6 million initial disability applications to estimate the effect of application costs on the composition of program participants.
        \item The dataset includes all initial disability applications filed between 2006 and 2015, and includes detailed information on the characteristics of applicants, the medical evidence they submit, and the outcomes of their applications.
        \item They use variables such as disability type and severity, age, education, and languages spoken to estimate the effect of application costs on the composition of program participants.
        \item importantly, study channels through which application costs affect the composition of program participants, use office wait times, processing times and staff counts to quantify congestation at neighborhood offices.
    \end{itemize}
\end{frame}
    
\begin{frame}{First Stage}
    \begin{figure}
        \centering
        \includegraphics[width=0.8\textwidth]{F1.png}
    \end{figure}
\end{frame}

\begin{frame}{Zip Codes}
    \begin{figure}
        \centering
        \includegraphics[width=0.8\textwidth]{F2.png}
    \end{figure}
\end{frame}

\begin{frame}{Empirical Strategy}
    \begin{itemize}
        \item The SSA office closing in a zip code is considered as a treatment. However, there is some selection in that. Instead, timing of closing is fairly random.
        \item The zip codes that experience current closings are treatment, and the zip codes that experience closing after 2 years are control. 
        \item 
    \end{itemize}
\end{frame}


\begin{frame}{Effect on Applications and Allowances}
    \begin{figure}
        \centering
        \includegraphics[width=0.8\textwidth]{F3.png}
    \end{figure}
    Larger decline in recipients than applicants implies that closings disproportionately
discourage applications by those who would have been allowed by SSA adjudicators if they had applied.
\end{frame}

\begin{frame}{Who is Screened Out?}
    \begin{figure}
        \centering
        \includegraphics[width=0.8\textwidth]{T2.png}
    \end{figure}
\end{frame}

\begin{frame}{Who is Screened Out?}
    \begin{figure}
        \centering
        \includegraphics[width=0.8\textwidth]{F4.png}
    \end{figure}
\end{frame}


\begin{frame}{Who is Screened Out?}
    \begin{figure}
        \centering
        \includegraphics[width=0.8\textwidth]{T3.png}
    \end{figure}
\end{frame}

\begin{frame}{Robustness I}
    \begin{figure}
        \centering
        \includegraphics[width=0.8\textwidth]{F5.png}
    \end{figure}
    For control (unaffected) zip
    codes, all $ D^\tau_{ct} $
    are set to zero. For treatment (closing) zip codes, the $ D^\tau_{ct} $ are equal to
    one when the quarter is $\tau$ quarters after (or before, if negative) the closing 

\end{frame}

\begin{frame}{Mechanisms}
    \begin{figure}
        \centering
        \includegraphics[width=0.8\textwidth]{F6.png}
    \end{figure}
\end{frame}

\begin{frame}{Mechanisms}
    \begin{figure}
        \centering
        \includegraphics[width=0.8\textwidth]{T4.png}
    \end{figure}
\end{frame}


\begin{frame}{Cost and Benefits of Closings}
    \begin{figure}
        \centering
        \includegraphics[width=0.8\textwidth]{T6.png}
    \end{figure}
    where $\gamma$ is coefficient of relative risk aversion.
\end{frame}


\begin{frame}{Welfare Calculations}
    \begin{figure}
        \centering
        \includegraphics[width=0.8\textwidth]{T7.png}
    \end{figure}
\end{frame}


\begin{frame}{Conclusion}
    \begin{itemize}
        \item Closings of SSA offices reduces the number of applicants by 10\% and the number of recipients by 16\%.
        \item These effects persistent even after 2 years of closing.
        \item The closings disproportionately discourage applications by applicants with lower education and pre-application earnings levels and moderately to severe conditions. 
        \item Importantly, this increase in hassles costs reduces both production and targeting efficiency.
        \item The field office closings exacerbate
        the very inequality that disability programs are intended to mitigate.
    \end{itemize}
\end{frame}
    
\end{document}
