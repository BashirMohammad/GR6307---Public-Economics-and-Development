% This file creates beamer slides for the paper Balboni et al 2024 FiFirm Adaptation in Production Networks: Evidence from Extreme Weather Events in Pakistan

\documentclass{beamer}

\usepackage[utf8]{inputenc} % Ensure proper encoding
\usepackage{graphicx} % For including images
\usepackage{hyperref} % For hyperlinks
\usepackage{amsmath,amsfonts,amssymb} % For mathematical symbols and equations

\setbeamertemplate{footline}[frame number] % Add page numbers to each slide

\begin{document}

\title{Firm Adaptation in Production Networks: Evidence from Extreme Weather Events in Pakistan}
\author{Clare Balboni, Johannes Boehm and Mazhar Waseem}

\institute{Muhammad Bashir, BSCD Lab}
\date{\today}

\frame{\titlepage}

\begin{frame}{Introduction}
\begin{itemize}
    \item 
    
\end{itemize}

\end{frame}

\begin{frame}[allowframebreaks]{Thoughts}
\begin{itemize}
    
\item Forward looking adaptation hard to show empirically given data limitation but they do reasonbly good job. Questions:
\begin{itemize}
    \item They disentangle forward looking adaptation from mechinical adaptation by showing that they are also shifting away from other non-flooded suppliers who have flood risk? But can this also be mediated by the mechinical adaptation since now you employ different routes or you just choose a group of suppliers instead of each supplier separately? It is hard to show this since they do not observe goods supplied by a given supplier but just the PKR amount of purchases from that supplier. 
    \item The adaptation is very quick? Does this mean something for adaptation channels? Could it be that same old suppliers are helping their purchasers source from other suppliers?
    \item What is evidence in the literature in general on frictions to adaptation or just general optimizing frictions? Kleven and Waseem (2013) optimizing frictions? 
    \item External Validity: Could it be that these Pakistani manufacturing firms are producing very basic goods and therefore it is easy to adjust? If you are producing more complex goods, it is harder to adjust as you would need very specific suppliers i.,e case of transistors in global supply chain disruptions?
    \item The mechnical adaptation also has many channels such as it could be just that your old supplier cannot supply or that prices and economic environment has changed and hence you re-optimize?
    \item People also asked how does a flood shock compare to other shocks such as financial shocks? 
    \item Similalry, where is this adjustment coming from? Are they trading off resilience to some other shocks in order to be more resilient to flood shocks? 
    \item Technically firms could also search for new buyers or customers in response to shocks to places where they make sales? They do not study this!
    \item Similarly, what are effects on the firms up the supply chains? Low power to study this!
    \item What is governmet's response in terms of investments in flood shock resilient infrastructure?
    \item It would good to study characteristics of new destinations and new suppliers.
    \item 
    
    
\end{itemize}

\end{itemize}
\end{frame}

\end{document}
