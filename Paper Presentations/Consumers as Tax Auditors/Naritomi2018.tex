\documentclass{beamer}

\title{Consumers as Tax Auditors}
\author{Joana Naritomi}
\date{\today}

\begin{document}

\frame{\titlepage}

\begin{frame}
\frametitle{Introduction}
\begin{itemize}
\item Access to third-party information and paper trails believed to be critical for administration of modern tax systems.
\item Despite collusion opportunities, how can government still effectively utilize this paper trail generated by economic activities. 
\item We discussed in this class that third-party reporting leads to substantial decrease in evasion  (Gordon \& Li 2009;
Kleven et al. 2016).
\item Therefore extensive margin effect of paper trail is known, but what are the mechanisms and how it avoids collusion among informal transcting parties.  

\end{itemize}
\end{frame}

\begin{frame}
\frametitle{This paper}
\begin{itemize}
    \item This paper exploits quasi-experimental variation and unique administrative data on firms and consumers from an anti-tax evasion program in Sao Paulo, Brazil – Nota Fiscal Paulista (NFP) – that created monetary rewards for consumers to ensure that firms report final sales transactions. T
    \item The program provides tax rebates and
monthly lottery prizes for consumers who ask for receipts, and establishes a direct communication
channel between the tax authority and consumers through an online account system, where consumers can verify receipts reported by firms and can act as whistle-blowers by filing complaints.
\end{itemize}
\end{frame}

\begin{frame}{Policy Background}
\begin{itemize}
    \item This policy was designed to solve last mile problem in paper trail mechanism of VAT. 
    \item When two firms transact, both of them report to tax authority. However, when final firm sells good to consumer, usually consumer does not report price and quantity to the tax authority. 
    \item However, firms and customers can also collude. Very common in restaurants and other small business! 
\end{itemize}
\end{frame}

\begin{frame}{Trade offs}
\begin{itemize}
    \item SS can be implemented under weaker assumptions. 
    \item What it actually identifies is credible and transparent. 
    \item We can always use SS even when we are not exactly sure about the model that generates this behaviour. 
    \item But you always need a formula for every economic question. It does not capture every aspect of the model. 
    \item This also cannot help with out of sample predictions in comparison to structural models.
\end{itemize}    
\end{frame}

\begin{frame}
\frametitle{Methodology}
\begin{itemize}
    \item Framework of the sufficient-statistic approach.
    \item Key formulas and models.
\end{itemize}
\end{frame}

\begin{frame}
\frametitle{Applications}
\begin{itemize}
    \item Application in income taxation.
    \item Application in social insurance.
\end{itemize}
\end{frame}

\begin{frame}
\frametitle{Behavioral Models}
\begin{itemize}
    \item Application to behavioral economics.
\end{itemize}
\end{frame}

\begin{frame}
\frametitle{Broader Implications}
\begin{itemize}
    \item Potential applications in various economic fields.
\end{itemize}
\end{frame}

\begin{frame}
\frametitle{Conclusion}
\begin{itemize}
    \item Main findings and contributions.
    \item Importance in policy analysis.
\end{itemize}
\end{frame}

\begin{frame}
\frametitle{References}
% Add your references here
\end{frame}

\end{document}
