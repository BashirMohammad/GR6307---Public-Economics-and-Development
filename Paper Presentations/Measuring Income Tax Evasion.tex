
\documentclass{beamer}
\usepackage{natbib}

\title{Measuring Income Tax Evasion Using Bank Credit: Evidence from Greece Get access Arrow by \citet{artavanis2016measuring}}
\author{}
\date{\today}
\begin{document}

\frame{\titlepage}

\begin{frame}
\frametitle{Some Odd Patterns in Bank Loans}
\begin{figure}
    \centering
    \includegraphics[width=\textwidth]{Paper Presentations/Table1.png}
\end{figure}
\end{frame}


\begin{frame}
\frametitle{Context }
\begin{itemize}
    \item Use loan decisions data from a big Greek Bank.
    \item After algorithm assigns the scores, bank managers make decisions after considering soft information such as potential real income.
    \item Data is universe of credit applications including those rejected.
    \item Tax data includes total income reported and total no of households at postal code and occupation level.
    \item Occupations: self-employed, wage workers,merchants and agriculture
    
\end{itemize}
\end{frame}

\begin{frame}{Methods}
\begin{itemize}
    \item Banks decide level of credit as a linear combination of their assessed true income, risk level and other area X individual soft skills.
    \[credit=\beta_1 Y^{True}+\Phi Risk+\Psi
    SOFT\]
    \item In data, you observe reported income $Y^R$. They use $credit$ to infer $Y^{True}$.
    \item Assume that 
    \[Y_{ij}^{True}= \begin{cases}
			\lambda_j Y_{ij}^{R}, & \text{if self-employed}\\
            Y_{ij}^R, & \text{if wage worker}
		 \end{cases}\]
   \item Use this to break credit equation to 
   \[credit_{ij}=\beta_{1j}Y^R{ij}(1-SE_i)+\beta_{2j}Y^R SE_i+f.e^{CreditGrade}+SOFT_{ij}\Psi
   \]
   $\lambda$ is ratio of the two betas. What are identifying assumptions of this equation?
\end{itemize}
\end{frame}

\begin{frame}{Results}
    \begin{figure}
        \centering
                \caption{Debt Capacity}
\includegraphics[height=\textheight,keepaspectratio]{Paper Presentations/R1.png}
    \end{figure}
\end{frame}

\begin{frame}{Results}
    \begin{figure}
        \centering
                \caption{Estimates}
\includegraphics[width=\textwidth,height=\textheight,keepaspectratio]{Paper Presentations/T2.png}
    \end{figure}
\end{frame}

\begin{frame}{Results}
    \begin{figure}
        \centering
                \caption{Mortgage Sample}
\includegraphics[width=\textwidth,height=\textheight,keepaspectratio]{Paper Presentations/T4.png}
    \end{figure}
\end{frame}

\begin{frame}
\frametitle{Results}
\begin{itemize}
    \item A $\lambda$ of 1.75 translates into 43\% evasion rate.
\end{itemize}
\end{frame}

\begin{frame}
\frametitle{Applications}
\begin{itemize}
    \item Application in income taxation.
    \item Application in social insurance.
\end{itemize}
\end{frame}

\begin{frame}
\frametitle{Behavioral Models}
\begin{itemize}
    \item Application to behavioral economics.
\end{itemize}
\end{frame}

\begin{frame}
\frametitle{Broader Implications}
\begin{itemize}
    \item Potential applications in various economic fields.
\end{itemize}
\end{frame}

\begin{frame}
\frametitle{Conclusion}
\begin{itemize}
    \item Main findings and contributions.
    \item Importance in policy analysis.
\end{itemize}
\end{frame}

\begin{frame}{Bibliography}
\bibliographystyle{plainnat}
\bibliography{Paper Presentations/bib}
\end{frame}

\end{document}
